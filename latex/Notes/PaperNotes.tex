\section{Paper Notes:}


% 1
\subsection{\citep{swinnen2005quantifying}}
the effects of the amplitude of temperature and the pre/post-shift temperature level have on the occurrence and length of a lag phase.\\
conclusion: the (intermediate) lag phase is influenced clearly bythe normal physiological temperature range, contrast to \citep{mellefont2003effect}'s result: independent of amplitude and normal physio-logical range, had little effect on the relative lagphase duration, but negative temperature has effects.\\


% 2
\subsection{On the lag phase and initial decline of microbial growth curves \citep{yates2007lag}}
*intro “daikan”\\
Various models have been proposed for the initial growing phase of bacteria \citep{baranyi1998comparison, baranyi1993non, baranyi1994dynamic, hills1994new, mckellar1997heterogeneous, mckellar2001development}. see review\citep{swinnen2004predictive}(\%3), quantifying lag phase helps prediction\citep{baty2004estimating} 


% 3
\subsection{Predictive modelling of the microbial lag phase: a review \citep{swinnen2004predictive}}
\subsubsection{Intro}
\begin{enumerate}
    \item primary models:
    \begin{enumerate}
        \item deterministic models: baranyi, gompertz
        \item stochastic models: buchanan
    \end{enumerate}
\end{enumerate}

\subsubsection{quantifying lag time}
\begin{enumerate}
    \item define the lag time:
    in my project use this\citep{buchanan1990mathematical} to define t\_lag when the third derivative equals 0. The illustration figure can be found in \citep{zwietering1992comparison}\\
    \item measuring lag time (population/individual)
    \begin{enumerate}
        \item pop: total viable count/ optical density (OD)
        \item individual: \(t\_lag = \tau\) (adaptation time) + DT (doubling time)
        \item OD is rather less accurate in counting the actual cell number in lag phase than plate count. The main problem with OD techniques is that the detection threshold typically corresponds to a bacterial concentration greater than 106 bacteria/ml\citep{begot1996recommendations}(haven't read(HR), cite from\citep{swinnen2005quantifying}). (\%1)
    \end{enumerate}
\end{enumerate}

\subsubsection{factors can influence the duration of the lag phase}
\begin{enumerate}
    \item the <EVN1>:
    temperature, pH..., <EVN1>\&<EVN2>, \(\Delta<EVN1>\) and the rate of the change (\(\Delta<EVN1> / \Delta t\))\\.
    \item the state of the cells
    \item the \textbf{variation} of the populations:
    \begin{enumerate}
        \item if the \textbf{inoculation size} is too small -> the variation between individual cells is higher -> longer lag phase
        \item if the cells are inoculated from the lag/station phase of <EVN1>, -> the variation is higher -> longer lag phase
    \end{enumerate}
\end{enumerate}
    
\subsubsection{existing modelling approaches}
<work(needed to adapt to <ENV2>)> = <rate><lag>
\begin{enumerate}
    \item 
\end{enumerate}


%4
\subsection{\citep{lenski1998evolution}}
(low probability citing)
hitchhick, linkage


%5
\subsection{The lag-phase during diauxic growth is a trade-off between fast adaptation and high growth rate \citep{chu2016lag}}

diauxic/bi-phasic phenomenon, may not suitable of inoculation, cause the diauxic lag may cause if the microorganisms don't allocate more energy into utilize the preferred sugar, which may have advantage of promoting the fitness of organisms, the cost of intraspecific competition will surpass the benefit of earlier preparation of transferring the use of sugar source \\
the explanation of the lag phase: \\
\begin{enumerate}
    \item time required to induce gene[17,18] -> doesn't seem to determining the length of the lag phase
    \item the lag phase is more likely under evolution control[8,19]
    \item unequal distribution of growth rate within population[4,18,21]
\end{enumerate} citation



%6
\subsection{On the duration of the microbial lag phase \citep{vermeersch2019duration}}:
main: what cause the arise of the lag phase, and what factors influence the duration of the lag phase
\subsubsection{the time bacteria needed to adapt to the new environment}
\subsubsection{the past the environment can influence the present}
activation of the respiration as the crucial factor for bacteria to escape the lag phase\\
the unsolved question: why the respiration can be the restrictive factor, there is an assumption that the ATP is the factor (more effective energy source maybe can explain it), there is phenomenon observed by (Perez-Samper et al. 2018) that the drop is more severe in cells that cannot respire. !? how many ways are there for bacteria to repress respiration?
\subsubsection{the variation within the individuals in the population}
it is to say that even in genetically identical population, some individuals adapt faster then some others. \\
original sentence for the first explanation: One simple explanation would be that the lag duration depends on the cell cycle stage a particular cell is in when the carbon source shift happens\\
biological noise: the variation maybe the more favorable strategy if the environment is changing frequently or unpredictable, so that when the environment is changed, there is always some most adaptive individuals can growth immediately. \\


%7
\subsection{Lag Phase Is a Dynamic, Organized, Adaptive, and Evolvable Period That Prepares Bacteria for Cell Division \citep{bertrand2019lag}}
more detail than \%6 \\




hints: you duoshao variation shi keyi bei wuzhong fenlei jieshi de you you duoshao shi bei huanjing jieshi de.\\

will the velocity of adaptation(the duration of the lag phase) influence the carbon gain of the bacteria? \\ 
% (https://onlinelibrary.wiley.com/doi/full/10.1111/btp.12993#btp12993-bib-0053)
\\


% 7
The mechanism of the metabolism happens in the cell when encountering new environment \citep{shimizu2013metabolic}























==================\\ qian bi ji
- Growth Rate and Generation Time of Bacteria, with Special Reference to Continuous Culture
- When is simple enough
- \\



- qian biye lunwen\\



**Natural laws** describe the behaviour of variables: they explain how the variable value(the quality/quantity properties of a variable) would change in corresponding to the other.

pay attention to **the scope of your law and variable** 

natural law deals with variables but pays attention to whether the scope of the laws match the scope of the observations

[in my research: data set (without specific other circumstance criteria) ](https://www.notion.so/9a9c5e460bd14affb77eb4d76eb30160)

you can observe the natural laws by collecting data and spotting a pattern

incomplete laws predict distributions

Chapter 2 - Distributions

Summary: Distributions reveal useful information, but the information is probabilistic.

parametric distribution

natural laws explain how the variable vary which makes distribution an important tool

July 8, 2021 Meeting

model selection: (popProject shoucang jia)\\
== qian wenzhang

can't understand how it infers that the MTE imply 

how could MTE predict the constrain(trade-off) between growth rate and K: as temperature

high growth rates, low life expectancies and low carrying capacity

Method: model fitting

Result: